\chapter{Závěr}
Cílem práce bylo analyzovat možnosti počítačového vidění v biomedicíně. Byla představena kapslová endoskopie, její využití a možnosti. Dále se práce zabývala představením počítačového vidění a jeho základními principy. Průzkum aktuálního stavu trhu sumarizoval již hotové řešení od výrobců kapslí, to bohužel vzhledem k uzavřenosti programů nepřineslo mnoho výsledků.

Praktická část se zabývala vývojem nového softwaru, který by byl alespoň částečně konkurence schopný a uměl detekovat krev, případně v budoucnu i další choroby. Vzorky pro testování byly dodány Fakultní nemocnicí Hradec Králové. 

Samotný vývoj software se ukázal jako časově náročný, ale obešel se bez výraznějších problémů, jelikož zvolené technologie byly autorovi dobře známé. S algoritmy byla daleko větší práce. Autor měl sice předchozí zkušenosti s počítačovým viděním, ale nebyly nikterak četné. Neocenitelnou pomoc pro vývoj algoritmů skýtaly konzultace s Dr. Orcanem Alparem.

Testování ukázalo, že software je schopný detekovat krev ve střevech s vysokou přesností. Aby byly výsledky průraznější, bylo by vhodné v tomto projektu pokračovat i mimo tuto diplomovou práci a provést další, rozsáhlejší testy. Zatím je vývoj projektu na dobré cestě a myslím si, že výsledky jsou uspokojivé.