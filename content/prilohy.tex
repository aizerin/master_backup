\begin{flushleft}
\printbibliography[heading=bibintoc,title={Literatura}]
\end{flushleft}
\cleardoublepage{}

\newglossaryentry{glos:GUI}{name=GUI, description={Graphical user interface - uživatelské rozhraní}}
\newglossaryentry{glos:CV}{name=CV, description={Computer vision - počítačové vidění}}
\newglossaryentry{glos:EER}{name=EER, description={Equal error rate - přijatelná míra chyb}}
\newglossaryentry{glos:FAR}{name=FAR, description={False acceptance rate - míra špatně detekovaných objektů}}
\newglossaryentry{glos:FRR}{name=FRR, description={False rejection rate - míra chybně odmítnutých objektů}}
\newglossaryentry{glos:AVI}{name=AVI, description={Multimediální video formát}}
\newglossaryentry{glos:JPG}{name=JPG, description={Metoda komprese obrázku}}
\newglossaryentry{glos:JNA}{name=JNA, description={Java native access - funkcionalita pro volání nativních knihoven z Javy}}
\newglossaryentry{glos:CDI}{name=CDI, description={Contexts and Dependency Injection - standart pro vkládání závislostí a kontextu. Poprvé byl uveřejněn pro Javu EE 6}}
\newglossaryentry{glos:MVC}{name=MVC, description={Model-View-Controller - návrhový vzor}}
\newglossaryentry{glos:WCE}{name=WCE, description={Wireless capsule endoskopy - bezdrátová kapslová endoskopie}}
\newglossaryentry{glos:TP}{name=TP, description={True positivity - pozitivní anomálie}}
\newglossaryentry{glos:FP}{name=FP, description={False positivity - falešně pozitivní anomálie}}
\newglossaryentry{glos:TN}{name=TN, description={True negativity - nepozitivní anomálie}}
\newglossaryentry{glos:FN}{name=FN, description={False negativity - falešně nepozitivní anomálie}}
\newglossaryentry{glos:MDR}{name=MDR, description={Missed detection rate - stejné jako FRR}}
\newglossaryentry{glos:SE}{name=SE, description={Sensitivity - pravděpodobnost detekce}}
\newglossaryentry{glos:SP}{name=SP, description={Specificity - přesnost detekce}}
\newglossaryentry{glos:ROI}{name=ROI, description={Rectangle of interest - čtverec/oblast zájmu}}
{\setstretch{1.0}
	\printglossary[nonumberlist]
}


\appendix
\pagenumbering{Roman}\addcontentsline{toc}{part}{Přílohy}\thispagestyle{empty}  \renewcommand{\appendixname}{P\v{r}iloha}%%přílohy, číslování římskými

\part*{Přílohy} %% rename

\listoffigures

\listoftables

\chapter{Konfigurace algoritmů}
\label{sec:prilohaKonfig}
V této příloze je konfigurace algoritmů a její vysvětlení.
\begin{table}[h]
	\begin{adjustwidth}{-.5in}{-.5in} 
		\centering
		\begin{tabular}{|l|l|c|c|}
			\hline
			\multicolumn{1}{|c|}{\bf Konfigurace} & \multicolumn{1}{c|}{\bf Význam} & \bf Málé skvrny & \bf Velké skvrny \\ \hline
			highH & \begin{tabular}[c]{@{}l@{}}maximální hranice filtru pro H kanál\\ ve formátu HSV\end{tabular} & Y & Y \\ \hline
			lowH & \begin{tabular}[c]{@{}l@{}}minimální hranice filtru pro H kanál\\ ve formátu HSV\end{tabular} & Y & Y \\ \hline
			highS & \begin{tabular}[c]{@{}l@{}}maximální hranice filtru pro S kanál\\ ve formátu HSV\end{tabular} & Y & Y \\ \hline
			lowS & \begin{tabular}[c]{@{}l@{}}minimální hranice filtru pro S kanál\\ ve formátu HSV\end{tabular} & Y & Y \\ \hline
			highV & \begin{tabular}[c]{@{}l@{}}maximální hranice filtru pro V kanál\\ ve formátu HSV\end{tabular} & Y & Y \\ \hline
			lowV & \begin{tabular}[c]{@{}l@{}}minimální hranice filtru pro V kanál\\ ve formátu HSV\end{tabular} & Y & Y \\ \hline
			kernelDilate & velikost jádra pro dilataci & Y & Y \\ \hline
			minContourArea & minimální velikost spojité oblasti & Y & Y \\ \hline
			highHPost & \begin{tabular}[c]{@{}l@{}}maximální hranice zpřesňujícího filtru\\ pro H kanál ve formátu HSV\end{tabular} & Y & N \\ \hline
			lowHPost & \begin{tabular}[c]{@{}l@{}}minimální hranice filtru zpřesňujícího filtru\\ pro H kanál ve formátu HSV\end{tabular} & Y & N \\ \hline
			highSPost & \begin{tabular}[c]{@{}l@{}}maximální hranice zpřesňujícího filtru\\ pro S kanál ve formátu HSV\end{tabular} & Y & N \\ \hline
			lowSPost & \begin{tabular}[c]{@{}l@{}}minimální hranice filtru zpřesňujícího\\ pro S kanál ve formátu HSV\end{tabular} & Y & N \\ \hline
			highVPost & \begin{tabular}[c]{@{}l@{}}maximální hranice zpřesňujícího filtru\\ pro V kanál ve formátu HSV\end{tabular} & Y & N \\ \hline
			lowVPost & \begin{tabular}[c]{@{}l@{}}minimální hranice filtru zpřesňujícího filtru\\ pro V kanál ve formátu HSV\end{tabular} & Y & N \\ \hline
			blurKernelSize & velikost jádra pro rozmazání & Y & N \\ \hline
			cannyThreshMin & minimálníh hranice práhování pro canny & Y & N \\ \hline
			cannyThreshMax & maximální hranice práhování pro canny & Y & N \\ \hline
			apertureSize & velikost sobelova operátoru & Y & N \\ \hline
			maxCircleRadius & maximální velikost kružnice & Y & N \\ \hline
			crop & hodnota oříznutí & Y & Y \\ \hline
			acceptArea & minimální procentuální velikost & N & Y \\ \hline
		\end{tabular}
		\caption{Konfigurace algoritmů}
		\label{tab:cfg}
	\end{adjustwidth}
\end{table}

\chapter{Konkrétní konfigurace algoritmů}
\label{sec:prilohaHodnoty}
V této příloze jsou konkrétní hodnoty konfigurace algoritmů, se kterou byli testovány data. Konfigurace jsou pouze jako textový výstup, jelikož konfigurační XML soubory jsou příliš velké.

\noindent
\begin{flushleft}
\textbf{Detekce malých skvrn - první vzorek\\}
apertureSize-3;
cannyThreshMax-175;
lowSPost-195;
blurKernelSize-3;
lowH-8;
lowVPost-200;
lowHPost-8;
highSPost-255;
highH-10;
lowS-190;
lowV-167;
highVPost-210;
minContourArea-5;
cannyThreshMin-140;
highHPost-9;
maxCircleRadius-15;
kernelDilate-10;
highS-255;
highV-220;
crop-0.1;
---\\
\textbf{Detekce malých skvrn - Endocapsule\\}
apertureSize-3;
cannyThreshMax-145;
lowSPost-135;
blurKernelSize-3;
lowH-165;
lowVPost-75;
lowHPost-165;
highSPost-255;
highH-179;
lowS-135;
lowV-75;
highVPost-191;
minContourArea-5;
cannyThreshMin-130;
highHPost-179;
maxCircleRadius-15;
kernelDilate-10;
highS-255;
highV-191;
crop-0.1;
---\\
\textbf{Detekce malých skvrn - PillCam\\}
apertureSize-3;
cannyThreshMax-145;
lowSPost-190;
blurKernelSize-3;
lowH-8;
lowVPost-160;
lowHPost-8;
highSPost-255;
highH-11;
lowS-190;
lowV-150;
highVPost-220;
minContourArea-5;
cannyThreshMin-130;
highHPost-11;
maxCircleRadius-15;
kernelDilate-10;
highS-255;
highV-220;
crop-0.1;
---\\
\textbf{Detekce velkých skvrn - první vzorek\\}
lowV-167;
acceptArea-0.15;
minContourArea-5;
lowH-8;
kernelDilate-5;
highS-255;
highH-10;
lowS-190;
crop-0.1;
highV-220;
---\\
\textbf{Detekce velkých skvrn - Endocapsule\\}
lowV-75;
acceptArea-0.15;
minContourArea-5;
lowH-165;
kernelDilate-5;
highS-255;
highH-179;
lowS-135;
crop-0.1;
highV-191;
---\\
\textbf{Detekce velkých skvrn - PilLCam\\}
lowV-61;
acceptArea-0.15;
minContourArea-5;
lowH-6;
kernelDilate-5;
highS-255;
highH-9;
lowS-170;
crop-0.1;
highV-255;
\end{flushleft}