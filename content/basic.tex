\cleardoublepage{}~\thispagestyle{empty}\begin{center}\pagenumbering{roman}
	
	\textsf{\Large Univerzita Hradec Králové}
	
	\vspace{0.5em}
	\textsf{\Large Fakulta informatiky a managementu}
	
	\vspace*{1em}
	\textsf{\Large Katedra informatiky a kvantitativních metod }
	
	\vspace{15mm}
	\includegraphics[width=0.5\textwidth]{images/uhk}
	
	\vspace{10mm}
	\textsf{\LARGE DIPLOMOVÁ PRÁCE}
	
	\vspace{10mm}
	\textsf{\LARGE Návrh a realizace software pro analýzu biomedicínských dat}
	
	\vspace{10mm}
\end{center} 

\vspace{10mm}
\begin{description}
	\item [{{\large Autor:}}] \noindent \textsf{\large Lukáš Sulík}
	\item [{{\large Studijní obor:}}] \noindent \textsf{\large AI2}
	\item [{{\large Vedoucí práce:}}] \noindent \textsf{\large doc. Ing. Ondřej Krejcar, Ph.D.}
\end{description}
\vfill
\hspace{2.5em}\textsf{\large Hradec Králové} \hfill \textsf{\large 2015}
\clearpage{}

%{\small \thispagestyle{plain}\addcontentsline{toc}{chapter}{Abstrakt} }{\small \par}

\newpage{}\thispagestyle{plain}

{\small \setcounter{page}{2} % nastavení číslování stránek
	\ }{\small \par}

\noindent {\small \vfill{}
	% set next text to bottom
	~}{\small \par}

\subsubsection*{Prohlášení}

\noindent Prohlašuji, že jsem diplomovou práci zpracoval samostatně a s použitím uvedené literatury.

\bigskip{}
\noindent V Hradci Králové dne \today\hspace{10em} Podpis:

\clearpage{}

\newpage{}\thispagestyle{plain}

{\small %\setcounter{page}{3} % nastavení číslování stránek
	\ }{\small \par}

\noindent {\small \vfill{}
	% set next text to bottom
	~}{\small \par}

\subsubsection*{Poděkování}

\noindent Rád bych poděkoval vedoucímu diplomové práce doc. Ing. Ondřeji Krejcarovi, Ph.D. za ochotu a úsilí vést tuto práci. Dr. Orcanovi Alparovi za konzultace ohledně vyvíjených algoritmů. A v neposlední řadě své rodině a přátelům, kteří mě při studiu podporovali.

\clearpage{}

\newpage{}\thispagestyle{plain}

\subsubsection*{Anotace}

Práce v teoretické části představuje kapslovou endoskopii a její využití ve zdravotnictví a zároveň základy počítačového vidění, jeho úskalí a přednosti. A dále představuje i několik základních algoritmů, které se v práci používají a průzkum trhu, jenž analyzuje aktuální stav technologií. Praktická část je založena na vývoji nového programu pro kapslovou endoskopii a základních algoritmů pro detekci krvácení v tenkém střevě. Výsledkem práce je software na platformě Java, využívající knihovnu OpenCV, který dokáže detekovat krvácení v zažívacím traktu.

\subsubsection*{Annotation}

Theoretical part of the thesis presents capsule endoscopy and its use in healthcare and basics of computer vision, its pitfalls and advantages. Along with that, it is presented several fundamental algorithms that are used in the thesis and market research, which analyzes the current state of the art. Practical part is based on the development of a new program for capsule endoscopy and basic algorithms for detecting bleeding in the small intestine. The result is software based on the Java platform, using the OpenCV library and it can detect bleeding in the digestive tract.

\cleardoublepage{}

{\small %%%   place for the signature
	%%%                                         *********
}{\small \par}

\cleardoublepage{}\thispagestyle{empty}
\tableofcontents{}